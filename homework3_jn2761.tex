\PassOptionsToPackage{unicode=true}{hyperref} % options for packages loaded elsewhere
\PassOptionsToPackage{hyphens}{url}
%
\documentclass[]{article}
\usepackage{lmodern}
\usepackage{amssymb,amsmath}
\usepackage{ifxetex,ifluatex}
\usepackage{fixltx2e} % provides \textsubscript
\ifnum 0\ifxetex 1\fi\ifluatex 1\fi=0 % if pdftex
  \usepackage[T1]{fontenc}
  \usepackage[utf8]{inputenc}
  \usepackage{textcomp} % provides euro and other symbols
\else % if luatex or xelatex
  \usepackage{unicode-math}
  \defaultfontfeatures{Ligatures=TeX,Scale=MatchLowercase}
\fi
% use upquote if available, for straight quotes in verbatim environments
\IfFileExists{upquote.sty}{\usepackage{upquote}}{}
% use microtype if available
\IfFileExists{microtype.sty}{%
\usepackage[]{microtype}
\UseMicrotypeSet[protrusion]{basicmath} % disable protrusion for tt fonts
}{}
\IfFileExists{parskip.sty}{%
\usepackage{parskip}
}{% else
\setlength{\parindent}{0pt}
\setlength{\parskip}{6pt plus 2pt minus 1pt}
}
\usepackage{hyperref}
\hypersetup{
            pdftitle={homework3\_jn2761},
            pdfauthor={jiaying Ning},
            pdfborder={0 0 0},
            breaklinks=true}
\urlstyle{same}  % don't use monospace font for urls
\usepackage[margin=1in]{geometry}
\usepackage{color}
\usepackage{fancyvrb}
\newcommand{\VerbBar}{|}
\newcommand{\VERB}{\Verb[commandchars=\\\{\}]}
\DefineVerbatimEnvironment{Highlighting}{Verbatim}{commandchars=\\\{\}}
% Add ',fontsize=\small' for more characters per line
\usepackage{framed}
\definecolor{shadecolor}{RGB}{248,248,248}
\newenvironment{Shaded}{\begin{snugshade}}{\end{snugshade}}
\newcommand{\AlertTok}[1]{\textcolor[rgb]{0.94,0.16,0.16}{#1}}
\newcommand{\AnnotationTok}[1]{\textcolor[rgb]{0.56,0.35,0.01}{\textbf{\textit{#1}}}}
\newcommand{\AttributeTok}[1]{\textcolor[rgb]{0.77,0.63,0.00}{#1}}
\newcommand{\BaseNTok}[1]{\textcolor[rgb]{0.00,0.00,0.81}{#1}}
\newcommand{\BuiltInTok}[1]{#1}
\newcommand{\CharTok}[1]{\textcolor[rgb]{0.31,0.60,0.02}{#1}}
\newcommand{\CommentTok}[1]{\textcolor[rgb]{0.56,0.35,0.01}{\textit{#1}}}
\newcommand{\CommentVarTok}[1]{\textcolor[rgb]{0.56,0.35,0.01}{\textbf{\textit{#1}}}}
\newcommand{\ConstantTok}[1]{\textcolor[rgb]{0.00,0.00,0.00}{#1}}
\newcommand{\ControlFlowTok}[1]{\textcolor[rgb]{0.13,0.29,0.53}{\textbf{#1}}}
\newcommand{\DataTypeTok}[1]{\textcolor[rgb]{0.13,0.29,0.53}{#1}}
\newcommand{\DecValTok}[1]{\textcolor[rgb]{0.00,0.00,0.81}{#1}}
\newcommand{\DocumentationTok}[1]{\textcolor[rgb]{0.56,0.35,0.01}{\textbf{\textit{#1}}}}
\newcommand{\ErrorTok}[1]{\textcolor[rgb]{0.64,0.00,0.00}{\textbf{#1}}}
\newcommand{\ExtensionTok}[1]{#1}
\newcommand{\FloatTok}[1]{\textcolor[rgb]{0.00,0.00,0.81}{#1}}
\newcommand{\FunctionTok}[1]{\textcolor[rgb]{0.00,0.00,0.00}{#1}}
\newcommand{\ImportTok}[1]{#1}
\newcommand{\InformationTok}[1]{\textcolor[rgb]{0.56,0.35,0.01}{\textbf{\textit{#1}}}}
\newcommand{\KeywordTok}[1]{\textcolor[rgb]{0.13,0.29,0.53}{\textbf{#1}}}
\newcommand{\NormalTok}[1]{#1}
\newcommand{\OperatorTok}[1]{\textcolor[rgb]{0.81,0.36,0.00}{\textbf{#1}}}
\newcommand{\OtherTok}[1]{\textcolor[rgb]{0.56,0.35,0.01}{#1}}
\newcommand{\PreprocessorTok}[1]{\textcolor[rgb]{0.56,0.35,0.01}{\textit{#1}}}
\newcommand{\RegionMarkerTok}[1]{#1}
\newcommand{\SpecialCharTok}[1]{\textcolor[rgb]{0.00,0.00,0.00}{#1}}
\newcommand{\SpecialStringTok}[1]{\textcolor[rgb]{0.31,0.60,0.02}{#1}}
\newcommand{\StringTok}[1]{\textcolor[rgb]{0.31,0.60,0.02}{#1}}
\newcommand{\VariableTok}[1]{\textcolor[rgb]{0.00,0.00,0.00}{#1}}
\newcommand{\VerbatimStringTok}[1]{\textcolor[rgb]{0.31,0.60,0.02}{#1}}
\newcommand{\WarningTok}[1]{\textcolor[rgb]{0.56,0.35,0.01}{\textbf{\textit{#1}}}}
\usepackage{graphicx,grffile}
\makeatletter
\def\maxwidth{\ifdim\Gin@nat@width>\linewidth\linewidth\else\Gin@nat@width\fi}
\def\maxheight{\ifdim\Gin@nat@height>\textheight\textheight\else\Gin@nat@height\fi}
\makeatother
% Scale images if necessary, so that they will not overflow the page
% margins by default, and it is still possible to overwrite the defaults
% using explicit options in \includegraphics[width, height, ...]{}
\setkeys{Gin}{width=\maxwidth,height=\maxheight,keepaspectratio}
\setlength{\emergencystretch}{3em}  % prevent overfull lines
\providecommand{\tightlist}{%
  \setlength{\itemsep}{0pt}\setlength{\parskip}{0pt}}
\setcounter{secnumdepth}{0}
% Redefines (sub)paragraphs to behave more like sections
\ifx\paragraph\undefined\else
\let\oldparagraph\paragraph
\renewcommand{\paragraph}[1]{\oldparagraph{#1}\mbox{}}
\fi
\ifx\subparagraph\undefined\else
\let\oldsubparagraph\subparagraph
\renewcommand{\subparagraph}[1]{\oldsubparagraph{#1}\mbox{}}
\fi

% set default figure placement to htbp
\makeatletter
\def\fps@figure{htbp}
\makeatother


\title{homework3\_jn2761}
\author{jiaying Ning}
\date{10/18/2020}

\begin{document}
\maketitle

\begin{Shaded}
\begin{Highlighting}[]
\KeywordTok{rm}\NormalTok{(}\DataTypeTok{list=}\KeywordTok{ls}\NormalTok{())}
\KeywordTok{library}\NormalTok{(readxl)}
\KeywordTok{library}\NormalTok{(tidyverse)}
\end{Highlighting}
\end{Shaded}

\begin{verbatim}
## -- Attaching packages ---------------------------------------------------------------------------------- tidyverse 1.3.0 --
\end{verbatim}

\begin{verbatim}
## v ggplot2 3.3.2     v purrr   0.3.4
## v tibble  3.0.3     v dplyr   1.0.2
## v tidyr   1.1.2     v stringr 1.4.0
## v readr   1.3.1     v forcats 0.5.0
\end{verbatim}

\begin{verbatim}
## -- Conflicts ------------------------------------------------------------------------------------- tidyverse_conflicts() --
## x dplyr::filter() masks stats::filter()
## x dplyr::lag()    masks stats::lag()
\end{verbatim}

\hypertarget{problem-1}{%
\subsubsection{Problem 1}\label{problem-1}}

\begin{Shaded}
\begin{Highlighting}[]
\CommentTok{#Import Data }
\NormalTok{Exercise_df=}
\StringTok{   }\KeywordTok{read.csv}\NormalTok{(}\StringTok{"./data/Exercise.csv"}\NormalTok{)}
\end{Highlighting}
\end{Shaded}

A study was conducted over a six-month period at a local ambulatory
virology clinic. The goal was to test the effect of a structured
exercise program for overweight/obese, virally suppressed HIV positive
subjects on different parameters. A total of 36 individuals agreed to
participate in the intervention group (group 1) and another group of 36
individuals were selected as controls (group 0). The table below shows
descriptive statistics: mean(SD), median(Q1, Q3) to summarize the
Systolic Blood Pressure (SBP) variable by groups at baseline (pre), at 6
months follow-up (post) and also the absolute changes (Δ=Post-Pre). We
want to perform some tests to assess changes in SBP for the two groups
(within and between). For each question, make sure to state the formulae
for hypotheses, test-statistics, decision rules/p-values, and provide
interpretations in the context of the problem. Use a type I error of
0.05 for all tests. Note: The raw dataset `Exercise.csv' used to
generate this table can be found on Canvas.

\textbf{a) Perform appropriate tests to assess if the Systolic BP at 6
months is significantly different from the baseline values for each of
the groups:} \textbf{i) Intervention group (5p)}

since we are comparing I will use paired-t test to test the hypothesis.

\begin{itemize}
\tightlist
\item
  Null Hypothesis: There is no difference between baseline values and
  Systolic BP at 6 month for intervention group.
\item
  Alternative Hypothesis: There is difference between baseline values
  and Systolic BP at 6 month for intervention group
\end{itemize}

\begin{Shaded}
\begin{Highlighting}[]
\CommentTok{#creating intervention group dataframe}
\NormalTok{Exercise_df_intervention  =}
\NormalTok{Exercise_df }\OperatorTok
\StringTok{  }\KeywordTok{filter}\NormalTok{(Group}\OperatorTok{==}\DecValTok{1}\NormalTok{) }\OperatorTok
\StringTok{  }\KeywordTok{mutate}\NormalTok{(}\DataTypeTok{difference =}\NormalTok{ Systolic_POST }\OperatorTok{-}\StringTok{ }\NormalTok{Systolic_PRE)}
\end{Highlighting}
\end{Shaded}

\begin{Shaded}
\begin{Highlighting}[]
\CommentTok{#perform paired-t.test}
\NormalTok{sd_diff<-}\KeywordTok{sd}\NormalTok{(}\KeywordTok{pull}\NormalTok{(Exercise_df_intervention,difference))}
\NormalTok{test_weight<-}\KeywordTok{mean}\NormalTok{(}\KeywordTok{pull}\NormalTok{(Exercise_df_intervention,difference))}\OperatorTok{/}\NormalTok{(sd_diff}\OperatorTok{/}\KeywordTok{sqrt}\NormalTok{(}\KeywordTok{length}\NormalTok{(}\KeywordTok{pull}\NormalTok{(Exercise_df_intervention,difference))))}
\CommentTok{#perform paired-t.test using build-in function}
\KeywordTok{t.test}\NormalTok{(}\KeywordTok{pull}\NormalTok{(Exercise_df_intervention,Systolic_POST), }\KeywordTok{pull}\NormalTok{(Exercise_df_intervention,Systolic_PRE), }\DataTypeTok{paired=}\NormalTok{T, }\DataTypeTok{alternative=}\StringTok{"two.sided"}\NormalTok{)}
\end{Highlighting}
\end{Shaded}

\begin{verbatim}
## 
##  Paired t-test
## 
## data:  pull(Exercise_df_intervention, Systolic_POST) and pull(Exercise_df_intervention, Systolic_PRE)
## t = -2.9996, df = 35, p-value = 0.004953
## alternative hypothesis: true difference in means is not equal to 0
## 95 percent confidence interval:
##  -14.392386  -2.774281
## sample estimates:
## mean of the differences 
##               -8.583333
\end{verbatim}

\begin{itemize}
\tightlist
\item
  Using a type I error of 0.05, since we have p-value of 0.004953 for
  the current test, we reject the null hypothesis and conclude there is
  sufficient evidence to show that there is difference between baseline
  values and Systolic BP at 6 month for intervention group
\end{itemize}

\textbf{ii) Control group (5p)}

\begin{itemize}
\tightlist
\item
  Null Hypothesis: There is no difference between baseline values and
  Systolic BP at 6 month for control group.
\item
  Alternative Hypothesis: There is difference between baseline values
  and Systolic BP at 6 month for control group
\end{itemize}

\begin{Shaded}
\begin{Highlighting}[]
\CommentTok{#creating control group dataframe}
\NormalTok{Exercise_df_control  =}
\NormalTok{Exercise_df }\OperatorTok
\StringTok{  }\KeywordTok{filter}\NormalTok{(Group}\OperatorTok{==}\DecValTok{0}\NormalTok{) }\OperatorTok
\StringTok{  }\KeywordTok{mutate}\NormalTok{(}\DataTypeTok{difference =}\NormalTok{ Systolic_POST }\OperatorTok{-}\StringTok{ }\NormalTok{Systolic_PRE)}
\end{Highlighting}
\end{Shaded}

\begin{Shaded}
\begin{Highlighting}[]
\CommentTok{#perform paired-t.test }
\NormalTok{sd_diff<-}\KeywordTok{sd}\NormalTok{(}\KeywordTok{pull}\NormalTok{(Exercise_df_control,difference))}
\NormalTok{test_weight<-}\KeywordTok{mean}\NormalTok{(}\KeywordTok{pull}\NormalTok{(Exercise_df_control,difference))}\OperatorTok{/}\NormalTok{(sd_diff}\OperatorTok{/}\KeywordTok{sqrt}\NormalTok{(}\KeywordTok{length}\NormalTok{(}\KeywordTok{pull}\NormalTok{(Exercise_df_control,difference))))}
\CommentTok{#perform paired-t.test using build-in function}
\KeywordTok{t.test}\NormalTok{(}\KeywordTok{pull}\NormalTok{(Exercise_df_control,Systolic_POST), }\KeywordTok{pull}\NormalTok{(Exercise_df_control,Systolic_PRE), }\DataTypeTok{paired=}\NormalTok{T, }\DataTypeTok{alternative=}\StringTok{"two.sided"}\NormalTok{)}
\end{Highlighting}
\end{Shaded}

\begin{verbatim}
## 
##  Paired t-test
## 
## data:  pull(Exercise_df_control, Systolic_POST) and pull(Exercise_df_control, Systolic_PRE)
## t = -1.3502, df = 35, p-value = 0.1856
## alternative hypothesis: true difference in means is not equal to 0
## 95 percent confidence interval:
##  -8.345373  1.678706
## sample estimates:
## mean of the differences 
##               -3.333333
\end{verbatim}

\begin{itemize}
\tightlist
\item
  Using a type I error of 0.05, since we have p-value of 0.1856 for the
  current test, we fail to reject the null hypothesis and conclude there
  is no sufficient evidence to show that there is difference between
  baseline values and Systolic BP at 6 month for intervention group
\end{itemize}

\textbf{b) Now perform a test and provide the 95\% confidence interval
to assess the Systolic BP absolute changes between the two groups.
(12p)}

\begin{Shaded}
\begin{Highlighting}[]
\CommentTok{#calculate the absolute difference for both group}
\NormalTok{Exercise_df_control  =}
\NormalTok{Exercise_df_control }\OperatorTok
\StringTok{  }\KeywordTok{mutate}\NormalTok{(}\DataTypeTok{abs_difference =} \KeywordTok{abs}\NormalTok{(Systolic_POST }\OperatorTok{-}\StringTok{ }\NormalTok{Systolic_PRE))}

\NormalTok{Exercise_df_intervention  =}
\NormalTok{Exercise_df_intervention }\OperatorTok
\StringTok{  }\KeywordTok{mutate}\NormalTok{(}\DataTypeTok{abs_difference =} \KeywordTok{abs}\NormalTok{(Systolic_POST }\OperatorTok{-}\StringTok{ }\NormalTok{Systolic_PRE))}
\end{Highlighting}
\end{Shaded}

first we need to test for equality of variance

\begin{Shaded}
\begin{Highlighting}[]
\CommentTok{# Test the equality of variances}
\NormalTok{F_stats<-}\KeywordTok{sd}\NormalTok{(}\KeywordTok{pull}\NormalTok{(Exercise_df_control,abs_difference))}\OperatorTok{^}\DecValTok{2}\OperatorTok{/}\KeywordTok{sd}\NormalTok{(}\KeywordTok{pull}\NormalTok{(Exercise_df_intervention,abs_difference))}\OperatorTok{^}\DecValTok{2}
\NormalTok{F_stats}
\end{Highlighting}
\end{Shaded}

\begin{verbatim}
## [1] 0.6239293
\end{verbatim}

\begin{Shaded}
\begin{Highlighting}[]
\CommentTok{#since we know we have 36 observation, n-1=35 n-2=34}
\NormalTok{F_crit<-}\KeywordTok{qf}\NormalTok{(.}\DecValTok{975}\NormalTok{, }\DataTypeTok{df1=}\DecValTok{35}\NormalTok{, }\DataTypeTok{df2=}\DecValTok{34}\NormalTok{) }
\NormalTok{F_crit}
\end{Highlighting}
\end{Shaded}

\begin{verbatim}
## [1] 1.974435
\end{verbatim}

\begin{itemize}
\item
  when comparing the test statistics,Because F\_stats:0.6239293
  \textless{} F\_crit:1.9744352, we fail to reject H0 and conclude that
  the variances are not significantly different.
\item
  Then we can perform the two-sample t-test with equal variances
\end{itemize}

\begin{Shaded}
\begin{Highlighting}[]
\CommentTok{#X1bar-X2bar}
\NormalTok{mean_diff=}\KeywordTok{mean}\NormalTok{(}\KeywordTok{pull}\NormalTok{(Exercise_df_intervention,abs_difference))}\OperatorTok{-}\KeywordTok{mean}\NormalTok{(}\KeywordTok{pull}\NormalTok{(Exercise_df_control,abs_difference))}
\CommentTok{#pooled estimate of the variance}
\NormalTok{s_pool =}\StringTok{ }\KeywordTok{sqrt}\NormalTok{(((}\DecValTok{35}\OperatorTok{*}\KeywordTok{sd}\NormalTok{(}\KeywordTok{pull}\NormalTok{(Exercise_df_control,abs_difference))}\OperatorTok{^}\DecValTok{2}\NormalTok{)}\OperatorTok{+}\NormalTok{(}\DecValTok{35}\OperatorTok{*}\KeywordTok{sd}\NormalTok{(}\KeywordTok{pull}\NormalTok{(Exercise_df_intervention,abs_difference))}\OperatorTok{^}\DecValTok{2}\NormalTok{))}\OperatorTok{/}\NormalTok{(}\DecValTok{36}\OperatorTok{+}\DecValTok{36-2}\NormalTok{))}
\NormalTok{s_pool}
\end{Highlighting}
\end{Shaded}

\begin{verbatim}
## [1] 10.46231
\end{verbatim}

\begin{Shaded}
\begin{Highlighting}[]
\NormalTok{t_crit <-}\StringTok{ }\KeywordTok{qt}\NormalTok{(}\FloatTok{0.975}\NormalTok{,}\DecValTok{70}\NormalTok{) }
\NormalTok{t_crit}
\end{Highlighting}
\end{Shaded}

\begin{verbatim}
## [1] 1.994437
\end{verbatim}

\begin{Shaded}
\begin{Highlighting}[]
\NormalTok{lower=mean_diff}\OperatorTok{-}\NormalTok{(t_crit}\OperatorTok{*}\NormalTok{s_pool}\OperatorTok{*}\KeywordTok{sqrt}\NormalTok{((}\DecValTok{1}\OperatorTok{/}\DecValTok{36}\NormalTok{)}\OperatorTok{+}\NormalTok{(}\DecValTok{1}\OperatorTok{/}\DecValTok{36}\NormalTok{)))}
\NormalTok{upper=mean_diff}\OperatorTok{+}\NormalTok{(t_crit}\OperatorTok{*}\NormalTok{s_pool}\OperatorTok{*}\KeywordTok{sqrt}\NormalTok{((}\DecValTok{1}\OperatorTok{/}\DecValTok{36}\NormalTok{)}\OperatorTok{+}\NormalTok{(}\DecValTok{1}\OperatorTok{/}\DecValTok{36}\NormalTok{)))}
\NormalTok{lower}
\end{Highlighting}
\end{Shaded}

\begin{verbatim}
## [1] -1.723818
\end{verbatim}

\begin{Shaded}
\begin{Highlighting}[]
\NormalTok{upper}
\end{Highlighting}
\end{Shaded}

\begin{verbatim}
## [1] 8.112706
\end{verbatim}

\begin{itemize}
\tightlist
\item
  Therefore, the 95\% confidence interval to assess the Systolic BP
  absolute changes is (-1.7238175,8.1127064), Since the interval contain
  0, we fail to reject a null hypothesis and we do not have sufficient
  evidence to conclude that there is difference in the absolute changes
  between the two groups.
\end{itemize}

\textbf{c) What are the main underlying assumptions for the tests
performed in parts a) and b)? (3p)}

\begin{itemize}
\tightlist
\item
  The main assumpetion is that the the observed differences constutue a
  random sample from a normally distributed population of difference
\end{itemize}

\textbf{i) Use graphical displays to check the normality assumption and
discuss the findings. (3p)}

\begin{Shaded}
\begin{Highlighting}[]
\NormalTok{knitr}\OperatorTok{::}\NormalTok{opts_chunk}\OperatorTok{$}\KeywordTok{set}\NormalTok{(}
  \DataTypeTok{fig.width =} \DecValTok{6}\NormalTok{,}
    \DataTypeTok{fig.asp =} \FloatTok{.6}\NormalTok{,}
  \DataTypeTok{out.width =} \StringTok{"90%"}
\NormalTok{)}
\KeywordTok{theme_set}\NormalTok{(}\KeywordTok{theme_minimal}\NormalTok{()}\OperatorTok{+}\KeywordTok{theme}\NormalTok{(}\DataTypeTok{legend.position =} \StringTok{"bottom"}\NormalTok{))}

\NormalTok{display_control=}\KeywordTok{ggplot}\NormalTok{(Exercise_df_control, }\KeywordTok{aes}\NormalTok{(}\DataTypeTok{x=}\NormalTok{difference)) }\OperatorTok{+}
\StringTok{  }\KeywordTok{geom_histogram}\NormalTok{(}\DataTypeTok{fill=}\StringTok{"dark blue"}\NormalTok{)}\OperatorTok{+}
\StringTok{  }\KeywordTok{labs}\NormalTok{(}\DataTypeTok{titles =} \StringTok{"distribution of difference on control group"}\NormalTok{,}
       \DataTypeTok{x=} \StringTok{"pre-post difference"}\NormalTok{,}
       \DataTypeTok{y =} \StringTok{"frequency"}\NormalTok{  )}
\NormalTok{display_intervention=}\KeywordTok{ggplot}\NormalTok{(Exercise_df_intervention, }\KeywordTok{aes}\NormalTok{(}\DataTypeTok{x=}\NormalTok{difference)) }\OperatorTok{+}
\StringTok{  }\KeywordTok{geom_histogram}\NormalTok{(}\DataTypeTok{fill=}\StringTok{"salmon"}\NormalTok{) }\OperatorTok{+}\StringTok{ }
\KeywordTok{labs}\NormalTok{(}\DataTypeTok{titles =} \StringTok{"distribution of difference on intervention group"}\NormalTok{,}
       \DataTypeTok{x=} \StringTok{"pre-post difference"}\NormalTok{,}
       \DataTypeTok{y =} \StringTok{"frequency"}\NormalTok{  )}

\KeywordTok{library}\NormalTok{(patchwork)}

\NormalTok{display_control}\OperatorTok{+}\NormalTok{display_intervention}
\end{Highlighting}
\end{Shaded}

\begin{verbatim}
## `stat_bin()` using `bins = 30`. Pick better value with `binwidth`.
## `stat_bin()` using `bins = 30`. Pick better value with `binwidth`.
\end{verbatim}

\includegraphics{homework3_jn2761_files/figure-latex/unnamed-chunk-10-1.pdf}

\begin{Shaded}
\begin{Highlighting}[]
\NormalTok{display_control=}\KeywordTok{ggplot}\NormalTok{(Exercise_df_control, }\KeywordTok{aes}\NormalTok{(}\DataTypeTok{x=}\NormalTok{difference)) }\OperatorTok{+}
\StringTok{  }\KeywordTok{geom_density}\NormalTok{(}\DataTypeTok{fill=}\StringTok{"dark blue"}\NormalTok{,}\DataTypeTok{alpha=}\NormalTok{.}\DecValTok{5}\NormalTok{)}\OperatorTok{+}
\StringTok{  }\KeywordTok{labs}\NormalTok{(}\DataTypeTok{titles =} \StringTok{"distribution of difference on control group"}\NormalTok{,}
       \DataTypeTok{x=} \StringTok{"pre-post difference"}\NormalTok{,}
       \DataTypeTok{y =} \StringTok{"density"}\NormalTok{  )}
\NormalTok{display_intervention=}\KeywordTok{ggplot}\NormalTok{(Exercise_df_intervention, }\KeywordTok{aes}\NormalTok{(}\DataTypeTok{x=}\NormalTok{difference)) }\OperatorTok{+}
\StringTok{   }\KeywordTok{geom_density}\NormalTok{(}\DataTypeTok{fill=}\StringTok{"salmon "}\NormalTok{,}\DataTypeTok{alpha=}\NormalTok{.}\DecValTok{5}\NormalTok{) }\OperatorTok{+}\StringTok{ }
\KeywordTok{labs}\NormalTok{(}\DataTypeTok{titles =} \StringTok{"distribution of difference on intervention group"}\NormalTok{,}
       \DataTypeTok{x=} \StringTok{"pre-post difference"}\NormalTok{,}
       \DataTypeTok{y =} \StringTok{"density"}\NormalTok{  )}
\NormalTok{display_control}\OperatorTok{+}\NormalTok{display_intervention}
\end{Highlighting}
\end{Shaded}

\includegraphics{homework3_jn2761_files/figure-latex/unnamed-chunk-10-2.pdf}

\begin{itemize}
\tightlist
\item
  by looking at the two ggplot, the normality assumption seem to hold
  for control group but not so much for intervention group.(seems like
  there are two major peak in the distribution of difference in
  intervention group)
\end{itemize}

\textbf{ii) If normality is questionable, how does this affect the tests
validity and what are some possible remedies? (2p)}

\begin{itemize}
\tightlist
\item
  If normality is questionable, our conclusion made based on p-value
  will also be questionable, because we ware making our inference based
  on the assumpetion that the difference follow a normal distribution.
  Some of the remedies can be to increase sample size, because even if
  the populaiton distribution is not normal, when the sample size is big
  enough,the sampling distribution will be approximately normal. Also,
  we can use a more conservative p-value so that we are more cautious
  about the conclusion we made.
\end{itemize}

\hypertarget{problem-2}{%
\subsubsection{Problem 2}\label{problem-2}}

We have discussed the fact that we are not guaranteed to make the
correct decision by the process of hypothesis testing and there is
always some level of uncertainty in statistics. The two main errors that
we are trying to minimize/control are type I and type II. A type I error
occurs when we reject the null hypothesis 𝐻0, when 𝐻0 is true. When we
set the significance level at 5\%, we are saying that we will allow
ourselves to make a type I error less than 5\% of the time. In practice
we can only calculate this probability using a series of ``what if''
calculations, because we do not really know the truth. In this exercise
you learn how to create your own `true' scenario, simulate corresponding
data, and quantify the type I error over many repetitions.

\emph{Scenario: The average IQ score of Ivy League colleges is 120. We
will assume this to be the null hypothesis (true mean is 120) with a
standard deviation of 15 and a significance level of 5\%. For the
alternative hypothesis we will consider that the `true mean is less than
120'.}

Most of the time (95\%) when we generate a sample from the underlying
true distribution, we should fail to reject the null hypothesis since
the null hypothesis is true. Let us test it!

\textbf{a) Generate one random sample of size n=20 from the underlying
(null) true distribution. Calculate the test statistic, compare to the
critical value and report the conclusion: 1, if you reject 𝐻0 or 0, if
you fail to rejected 𝐻0. (5p) Hint: use rnorm(20, mean = 120, sd = 15)}

\begin{Shaded}
\begin{Highlighting}[]
\CommentTok{#generae a randome sample with size=20, mean=120, sd=15}
\KeywordTok{set.seed}\NormalTok{(}\DecValTok{2761}\NormalTok{)}
\NormalTok{rs1=}\KeywordTok{rnorm}\NormalTok{(}\DecValTok{20}\NormalTok{, }\DataTypeTok{mean =} \DecValTok{120}\NormalTok{, }\DataTypeTok{sd =} \DecValTok{15}\NormalTok{)}

\CommentTok{#calculate the criticle value}
\NormalTok{(}\KeywordTok{mean}\NormalTok{(rs1)}\OperatorTok{-}\DecValTok{120}\NormalTok{)}\OperatorTok{/}\NormalTok{(}\DecValTok{15}\OperatorTok{/}\KeywordTok{sqrt}\NormalTok{(}\DecValTok{20}\NormalTok{))}
\end{Highlighting}
\end{Shaded}

\begin{verbatim}
## [1] 0.6303571
\end{verbatim}

\begin{Shaded}
\begin{Highlighting}[]
\KeywordTok{qnorm}\NormalTok{(}\FloatTok{0.05}\NormalTok{, }\DataTypeTok{mean =} \DecValTok{0}\NormalTok{, }\DataTypeTok{sd =} \DecValTok{1}\NormalTok{)}
\end{Highlighting}
\end{Shaded}

\begin{verbatim}
## [1] -1.644854
\end{verbatim}

\begin{itemize}
\tightlist
\item
  The conclusion is 0 since our test statistic is 0.63 which is larger
  than -1.64, and we fail to reject the H0.
\end{itemize}

\textbf{b) Now generate 100 random samples of size n = 20 from the
underlying (null) true distribution and repeat the process in part (a)
for each sample (calculate the test statistic, compare to the critical
value, and record 1 or 0 based on criteria above). Report the percentage
of 1s and 0s respectively across the 100 samples. The percentage of 1s
represents the type I error. (7.5p)}

\begin{Shaded}
\begin{Highlighting}[]
\CommentTok{#creat two empty list}
\NormalTok{  l <-}\StringTok{ }\KeywordTok{list}\NormalTok{()}
\NormalTok{result=}\KeywordTok{list}\NormalTok{()}

\CommentTok{#generae 100 random sample with size=20, mean=120, sd=15 and store them in newly creatd list L, then record the decision of each random sample in list "result"}
\KeywordTok{set.seed}\NormalTok{(}\DecValTok{2761}\NormalTok{)}
\ControlFlowTok{for}\NormalTok{ (i }\ControlFlowTok{in} \DecValTok{1}\OperatorTok{:}\DecValTok{100}\NormalTok{)\{}
\NormalTok{l[i] <-}\StringTok{ }\KeywordTok{list}\NormalTok{(}\KeywordTok{rnorm}\NormalTok{(}\DecValTok{20}\NormalTok{, }\DataTypeTok{mean =} \DecValTok{120}\NormalTok{, }\DataTypeTok{sd =} \DecValTok{15}\NormalTok{))}
\NormalTok{\}}

\ControlFlowTok{for}\NormalTok{ (i }\ControlFlowTok{in} \DecValTok{1}\OperatorTok{:}\DecValTok{100}\NormalTok{)\{}
\NormalTok{result[i] =}\StringTok{ }\KeywordTok{ifelse}\NormalTok{((}\KeywordTok{mean}\NormalTok{(l[[i]])}\OperatorTok{-}\DecValTok{120}\NormalTok{)}\OperatorTok{/}\NormalTok{(}\DecValTok{15}\OperatorTok{/}\KeywordTok{sqrt}\NormalTok{(}\DecValTok{20}\NormalTok{)) }\OperatorTok{<}\StringTok{ }\FloatTok{-1.645}\NormalTok{,}\DecValTok{1}\NormalTok{,}\DecValTok{0}\NormalTok{)}
\NormalTok{\}}
\CommentTok{#convert list into columns}
\NormalTok{result_df =}\StringTok{ }\KeywordTok{c}\NormalTok{(}\KeywordTok{unlist}\NormalTok{(result))}

\CommentTok{#show the frequency of each decision}
\KeywordTok{table}\NormalTok{(result_df)}
\end{Highlighting}
\end{Shaded}

\begin{verbatim}
## result_df
##  0  1 
## 97  3
\end{verbatim}

\begin{itemize}
\tightlist
\item
  the percentage of 1s and 0s for the current cases with 100 sample size
  are 0.03 for 1s and 0.97 for 0s. The precentage of type I error is
  0.03 in the current cause.
\end{itemize}

\textbf{c) Now generate 1000 random samples of size n = 20 from the
underlying (null) true distribution, repeat the same process, and report
the percentage of 1s and 0s across the 1000 samples. (7.5p)}

\begin{Shaded}
\begin{Highlighting}[]
\CommentTok{#create two empty list}
\NormalTok{  l <-}\StringTok{ }\KeywordTok{list}\NormalTok{()}
\NormalTok{result=}\KeywordTok{list}\NormalTok{()}

\CommentTok{#generae 1000 random sample with size=20, mean=120, sd=15 and store them in newly creatd list L, then record the decision of each random sample in list "result"}
\KeywordTok{set.seed}\NormalTok{(}\DecValTok{2761}\NormalTok{)}
\ControlFlowTok{for}\NormalTok{ (i }\ControlFlowTok{in} \DecValTok{1}\OperatorTok{:}\DecValTok{1000}\NormalTok{)\{}
\NormalTok{l[i] <-}\StringTok{ }\KeywordTok{list}\NormalTok{(}\KeywordTok{rnorm}\NormalTok{(}\DecValTok{20}\NormalTok{, }\DataTypeTok{mean =} \DecValTok{120}\NormalTok{, }\DataTypeTok{sd =} \DecValTok{15}\NormalTok{))}
\NormalTok{result[i] =}\StringTok{ }\KeywordTok{ifelse}\NormalTok{((}\KeywordTok{mean}\NormalTok{(l[[i]])}\OperatorTok{-}\DecValTok{120}\NormalTok{)}\OperatorTok{/}\NormalTok{(}\DecValTok{15}\OperatorTok{/}\KeywordTok{sqrt}\NormalTok{(}\DecValTok{20}\NormalTok{)) }\OperatorTok{<}\StringTok{ }\NormalTok{(}\OperatorTok{-}\FloatTok{1.645}\NormalTok{),}\DecValTok{1}\NormalTok{,}\DecValTok{0}\NormalTok{)}
\NormalTok{\}}

\CommentTok{#convert list into columns}
\NormalTok{result_df =}\StringTok{ }\KeywordTok{c}\NormalTok{(}\KeywordTok{unlist}\NormalTok{(result))}

\CommentTok{#show the frequency of each decision}
\KeywordTok{table}\NormalTok{(result_df)}
\end{Highlighting}
\end{Shaded}

\begin{verbatim}
## result_df
##   0   1 
## 947  53
\end{verbatim}

\begin{itemize}
\tightlist
\item
  the percentage of 1s and 0s for the current cases are 0.053 for 1s and
  0.947 for 0s. The precentage of type I error is 0.053 in the current
  cases.*
\end{itemize}

\textbf{d) Final conclusions: compare the type I errors (percentage of
1s) from part b) and c). How do they compare to the level that we
initially imposed (i.e.~0.05)? Comment on your findings. (5p)}

\begin{itemize}
\tightlist
\item
  The frequency of Type I error I generated are 0.03 for 100 sample size
  and 0.053 for 1000 sample size. The type I error approches to the
  initially imposed type I error as we increase the sample size. From
  this obervation, my insight is that we need to be extra cautous when
  making inference for smaller sample size. Because when we are making
  inference we are assuming the underlying distribution is normal, but
  there are many natural variation that can confound the result when the
  sample size is small, however if the sample size is big enough, we can
  be more confidence when making inferenece.
\end{itemize}

Notes: For this problem you are encouraged to use R for all
calculations/simulations. You can follow the hints or feel free to use
other functions -- there are several ways to tackle these simulations.
You do not need to write the test statistics, critical values, etc., but
please include themain results (percentage of correct and incorrect
decisions) for each part and conclusions in the main homework document.
Make sure to comment your R code and don't forget to set the seed for
replicability.

\end{document}
