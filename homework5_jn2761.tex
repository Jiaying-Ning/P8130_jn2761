\PassOptionsToPackage{unicode=true}{hyperref} % options for packages loaded elsewhere
\PassOptionsToPackage{hyphens}{url}
%
\documentclass[]{article}
\usepackage{lmodern}
\usepackage{amssymb,amsmath}
\usepackage{ifxetex,ifluatex}
\usepackage{fixltx2e} % provides \textsubscript
\ifnum 0\ifxetex 1\fi\ifluatex 1\fi=0 % if pdftex
  \usepackage[T1]{fontenc}
  \usepackage[utf8]{inputenc}
  \usepackage{textcomp} % provides euro and other symbols
\else % if luatex or xelatex
  \usepackage{unicode-math}
  \defaultfontfeatures{Ligatures=TeX,Scale=MatchLowercase}
\fi
% use upquote if available, for straight quotes in verbatim environments
\IfFileExists{upquote.sty}{\usepackage{upquote}}{}
% use microtype if available
\IfFileExists{microtype.sty}{%
\usepackage[]{microtype}
\UseMicrotypeSet[protrusion]{basicmath} % disable protrusion for tt fonts
}{}
\IfFileExists{parskip.sty}{%
\usepackage{parskip}
}{% else
\setlength{\parindent}{0pt}
\setlength{\parskip}{6pt plus 2pt minus 1pt}
}
\usepackage{hyperref}
\hypersetup{
            pdftitle={homework5\_jn2761},
            pdfauthor={jiaying Ning},
            pdfborder={0 0 0},
            breaklinks=true}
\urlstyle{same}  % don't use monospace font for urls
\usepackage[margin=1in]{geometry}
\usepackage{color}
\usepackage{fancyvrb}
\newcommand{\VerbBar}{|}
\newcommand{\VERB}{\Verb[commandchars=\\\{\}]}
\DefineVerbatimEnvironment{Highlighting}{Verbatim}{commandchars=\\\{\}}
% Add ',fontsize=\small' for more characters per line
\usepackage{framed}
\definecolor{shadecolor}{RGB}{248,248,248}
\newenvironment{Shaded}{\begin{snugshade}}{\end{snugshade}}
\newcommand{\AlertTok}[1]{\textcolor[rgb]{0.94,0.16,0.16}{#1}}
\newcommand{\AnnotationTok}[1]{\textcolor[rgb]{0.56,0.35,0.01}{\textbf{\textit{#1}}}}
\newcommand{\AttributeTok}[1]{\textcolor[rgb]{0.77,0.63,0.00}{#1}}
\newcommand{\BaseNTok}[1]{\textcolor[rgb]{0.00,0.00,0.81}{#1}}
\newcommand{\BuiltInTok}[1]{#1}
\newcommand{\CharTok}[1]{\textcolor[rgb]{0.31,0.60,0.02}{#1}}
\newcommand{\CommentTok}[1]{\textcolor[rgb]{0.56,0.35,0.01}{\textit{#1}}}
\newcommand{\CommentVarTok}[1]{\textcolor[rgb]{0.56,0.35,0.01}{\textbf{\textit{#1}}}}
\newcommand{\ConstantTok}[1]{\textcolor[rgb]{0.00,0.00,0.00}{#1}}
\newcommand{\ControlFlowTok}[1]{\textcolor[rgb]{0.13,0.29,0.53}{\textbf{#1}}}
\newcommand{\DataTypeTok}[1]{\textcolor[rgb]{0.13,0.29,0.53}{#1}}
\newcommand{\DecValTok}[1]{\textcolor[rgb]{0.00,0.00,0.81}{#1}}
\newcommand{\DocumentationTok}[1]{\textcolor[rgb]{0.56,0.35,0.01}{\textbf{\textit{#1}}}}
\newcommand{\ErrorTok}[1]{\textcolor[rgb]{0.64,0.00,0.00}{\textbf{#1}}}
\newcommand{\ExtensionTok}[1]{#1}
\newcommand{\FloatTok}[1]{\textcolor[rgb]{0.00,0.00,0.81}{#1}}
\newcommand{\FunctionTok}[1]{\textcolor[rgb]{0.00,0.00,0.00}{#1}}
\newcommand{\ImportTok}[1]{#1}
\newcommand{\InformationTok}[1]{\textcolor[rgb]{0.56,0.35,0.01}{\textbf{\textit{#1}}}}
\newcommand{\KeywordTok}[1]{\textcolor[rgb]{0.13,0.29,0.53}{\textbf{#1}}}
\newcommand{\NormalTok}[1]{#1}
\newcommand{\OperatorTok}[1]{\textcolor[rgb]{0.81,0.36,0.00}{\textbf{#1}}}
\newcommand{\OtherTok}[1]{\textcolor[rgb]{0.56,0.35,0.01}{#1}}
\newcommand{\PreprocessorTok}[1]{\textcolor[rgb]{0.56,0.35,0.01}{\textit{#1}}}
\newcommand{\RegionMarkerTok}[1]{#1}
\newcommand{\SpecialCharTok}[1]{\textcolor[rgb]{0.00,0.00,0.00}{#1}}
\newcommand{\SpecialStringTok}[1]{\textcolor[rgb]{0.31,0.60,0.02}{#1}}
\newcommand{\StringTok}[1]{\textcolor[rgb]{0.31,0.60,0.02}{#1}}
\newcommand{\VariableTok}[1]{\textcolor[rgb]{0.00,0.00,0.00}{#1}}
\newcommand{\VerbatimStringTok}[1]{\textcolor[rgb]{0.31,0.60,0.02}{#1}}
\newcommand{\WarningTok}[1]{\textcolor[rgb]{0.56,0.35,0.01}{\textbf{\textit{#1}}}}
\usepackage{graphicx,grffile}
\makeatletter
\def\maxwidth{\ifdim\Gin@nat@width>\linewidth\linewidth\else\Gin@nat@width\fi}
\def\maxheight{\ifdim\Gin@nat@height>\textheight\textheight\else\Gin@nat@height\fi}
\makeatother
% Scale images if necessary, so that they will not overflow the page
% margins by default, and it is still possible to overwrite the defaults
% using explicit options in \includegraphics[width, height, ...]{}
\setkeys{Gin}{width=\maxwidth,height=\maxheight,keepaspectratio}
\setlength{\emergencystretch}{3em}  % prevent overfull lines
\providecommand{\tightlist}{%
  \setlength{\itemsep}{0pt}\setlength{\parskip}{0pt}}
\setcounter{secnumdepth}{0}
% Redefines (sub)paragraphs to behave more like sections
\ifx\paragraph\undefined\else
\let\oldparagraph\paragraph
\renewcommand{\paragraph}[1]{\oldparagraph{#1}\mbox{}}
\fi
\ifx\subparagraph\undefined\else
\let\oldsubparagraph\subparagraph
\renewcommand{\subparagraph}[1]{\oldsubparagraph{#1}\mbox{}}
\fi

% set default figure placement to htbp
\makeatletter
\def\fps@figure{htbp}
\makeatother


\title{homework5\_jn2761}
\author{jiaying Ning}
\date{11/19/2020}

\begin{document}
\maketitle

\begin{Shaded}
\begin{Highlighting}[]
\CommentTok{#Loading Packages}
\KeywordTok{rm}\NormalTok{(}\DataTypeTok{list=}\KeywordTok{ls}\NormalTok{())}
\KeywordTok{library}\NormalTok{(readxl)}
\KeywordTok{library}\NormalTok{(tidyverse)}
\end{Highlighting}
\end{Shaded}

\begin{verbatim}
## -- Attaching packages --------------------------------------- tidyverse 1.3.0 --
\end{verbatim}

\begin{verbatim}
## v ggplot2 3.3.2     v purrr   0.3.4
## v tibble  3.0.4     v dplyr   1.0.2
## v tidyr   1.1.2     v stringr 1.4.0
## v readr   1.3.1     v forcats 0.5.0
\end{verbatim}

\begin{verbatim}
## -- Conflicts ------------------------------------------ tidyverse_conflicts() --
## x dplyr::filter() masks stats::filter()
## x dplyr::lag()    masks stats::lag()
\end{verbatim}

\begin{Shaded}
\begin{Highlighting}[]
\KeywordTok{library}\NormalTok{(broom)}
\KeywordTok{library}\NormalTok{(dplyr)}
\end{Highlighting}
\end{Shaded}

\hypertarget{problem-1}{%
\subsection{Problem 1}\label{problem-1}}

\begin{Shaded}
\begin{Highlighting}[]
\CommentTok{#Import Data }
\NormalTok{antibodies_df=}
\StringTok{   }\KeywordTok{read.csv}\NormalTok{(}\StringTok{"./data/Antibodies.csv"}\NormalTok{) }\OperatorTok
\StringTok{  }\KeywordTok{na.omit}\NormalTok{() }
\end{Highlighting}
\end{Shaded}

\begin{Shaded}
\begin{Highlighting}[]
\NormalTok{Normal =}\StringTok{ }\NormalTok{antibodies_df }\OperatorTok
\StringTok{  }\KeywordTok{filter}\NormalTok{(Smell }\OperatorTok{==}\StringTok{ "Normal"}\NormalTok{)}\OperatorTok
\StringTok{  }\KeywordTok{mutate}\NormalTok{(}\DataTypeTok{Antibody_IgM =} \KeywordTok{sort}\NormalTok{(Antibody_IgM, }\DataTypeTok{decreasing =} \OtherTok{FALSE}\NormalTok{))}

\NormalTok{ALtered =}\StringTok{ }\NormalTok{antibodies_df }\OperatorTok
\StringTok{  }\KeywordTok{filter}\NormalTok{(Smell }\OperatorTok{==}\StringTok{ "Altered"}\NormalTok{) }\OperatorTok
\StringTok{  }\KeywordTok{mutate}\NormalTok{(}\DataTypeTok{Antibody_IgM =} \KeywordTok{sort}\NormalTok{(Antibody_IgM, }\DataTypeTok{decreasing =} \OtherTok{FALSE}\NormalTok{))}
\end{Highlighting}
\end{Shaded}

\hypertarget{non-parametric-test-for-assessing-the-difference-in-ig-m-levels-between-the-two-groups}{%
\subsubsection{non-parametric test for assessing the difference in Ig-M
levels between the two
groups}\label{non-parametric-test-for-assessing-the-difference-in-ig-m-levels-between-the-two-groups}}

\begin{itemize}
\tightlist
\item
  The non parametric test I will be using is the Non-parametric
  Wilcoxon-Rank Sum test,Since there are ties in our dataset, we will
  need to use adjust term in our calculation.
\end{itemize}

\textbf{Hypothesis}

\begin{itemize}
\item
  H0:the median Ig\_M levels are equal for both altered smell group and
  Normal smell group
\item
  H1:the median Ig\_M levels are not equal for both altered smell group
  and Normal smell group
\end{itemize}

\textbf{calculations}

\begin{Shaded}
\begin{Highlighting}[]
\NormalTok{res=}\KeywordTok{wilcox.test}\NormalTok{(Normal}\OperatorTok{$}\NormalTok{Antibody_IgM, ALtered}\OperatorTok{$}\NormalTok{Antibody_IgM, }\DataTypeTok{mu=}\DecValTok{0}\NormalTok{)}
\NormalTok{test_stats_df=}\KeywordTok{tidy}\NormalTok{(}\KeywordTok{wilcox.test}\NormalTok{(Normal}\OperatorTok{$}\NormalTok{Antibody_IgM, ALtered}\OperatorTok{$}\NormalTok{Antibody_IgM, }\DataTypeTok{mu=}\DecValTok{0}\NormalTok{))}
\NormalTok{res}
\end{Highlighting}
\end{Shaded}

\begin{verbatim}
## 
##  Wilcoxon rank sum test with continuity correction
## 
## data:  Normal$Antibody_IgM and ALtered$Antibody_IgM
## W = 5836, p-value = 0.01406
## alternative hypothesis: true location shift is not equal to 0
\end{verbatim}

\begin{Shaded}
\begin{Highlighting}[]
\NormalTok{w=res}\OperatorTok{$}\NormalTok{statistic}\OperatorTok{+}\StringTok{ }\DecValTok{81}\OperatorTok{*}\NormalTok{(}\DecValTok{81}\OperatorTok{+}\DecValTok{1}\NormalTok{)}\OperatorTok{/}\DecValTok{2}
\end{Highlighting}
\end{Shaded}

\textbf{Test Stats}

\begin{itemize}
\tightlist
\item
  W=9157
\item
  p-value=0.0140605
\end{itemize}

\textbf{Conclusion}

\begin{itemize}
\tightlist
\item
  Using a 0.05 significance level, since we have p-value less than
  0.05,, we reject H0 and conclude that the Normal smell and Altered
  Smell have significantly different median Ig\_M levels.
\end{itemize}

\hypertarget{problem-3}{%
\subsection{Problem 3}\label{problem-3}}

\hypertarget{part-1}{%
\subsubsection{part 1}\label{part-1}}

\begin{Shaded}
\begin{Highlighting}[]
\NormalTok{GPA_df=}
\StringTok{   }\KeywordTok{read.csv}\NormalTok{(}\StringTok{"./data/GPA.csv"}\NormalTok{)}
\end{Highlighting}
\end{Shaded}

\begin{Shaded}
\begin{Highlighting}[]
\NormalTok{GPA_df }\OperatorTok\StringTok{ }
\StringTok{  }\KeywordTok{ggplot}\NormalTok{(}\KeywordTok{aes}\NormalTok{(ACT, GPA)) }\OperatorTok{+}\StringTok{ }\KeywordTok{geom_point}\NormalTok{(}\DataTypeTok{color=}\StringTok{'black'}\NormalTok{)  }\OperatorTok{+}
\StringTok{  }\KeywordTok{geom_smooth}\NormalTok{(}\DataTypeTok{method=}\StringTok{'lm'}\NormalTok{, }\DataTypeTok{se=}\OtherTok{TRUE}\NormalTok{,}\DataTypeTok{color=}\StringTok{"pink"}\NormalTok{) }
\end{Highlighting}
\end{Shaded}

\begin{verbatim}
## `geom_smooth()` using formula 'y ~ x'
\end{verbatim}

\includegraphics{homework5_jn2761_files/figure-latex/unnamed-chunk-7-1.pdf}

\textbf{Hypothesis}

\begin{itemize}
\item
  H0:There is no linear association exists between student's ACT score
  (X) and GPA at the end of the freshman year (Y).beta1=0
\item
  H1:There is linear association exists between student's ACT score (X)
  and GPA at the end of the freshman year (Y).beta1!=0
\end{itemize}

\textbf{Calculations}

For the current test, we use the t-test to test whether or not the slope
between ACT and GPA is significantly different from 0, if it is, we
conclude that there is a linear association exists.

\[ (beta1-0)/se(beta1) \]

\begin{Shaded}
\begin{Highlighting}[]
\NormalTok{GPAlm=}\KeywordTok{lm}\NormalTok{(GPA}\OperatorTok{~}\NormalTok{ACT,}\DataTypeTok{data=}\NormalTok{GPA_df)}
\end{Highlighting}
\end{Shaded}

\begin{Shaded}
\begin{Highlighting}[]
\KeywordTok{summary}\NormalTok{(GPAlm)}
\end{Highlighting}
\end{Shaded}

\begin{verbatim}
## 
## Call:
## lm(formula = GPA ~ ACT, data = GPA_df)
## 
## Residuals:
##      Min       1Q   Median       3Q      Max 
## -2.74004 -0.33827  0.04062  0.44064  1.22737 
## 
## Coefficients:
##             Estimate Std. Error t value Pr(>|t|)    
## (Intercept)  2.11405    0.32089   6.588  1.3e-09 ***
## ACT          0.03883    0.01277   3.040  0.00292 ** 
## ---
## Signif. codes:  0 '***' 0.001 '**' 0.01 '*' 0.05 '.' 0.1 ' ' 1
## 
## Residual standard error: 0.6231 on 118 degrees of freedom
## Multiple R-squared:  0.07262,    Adjusted R-squared:  0.06476 
## F-statistic:  9.24 on 1 and 118 DF,  p-value: 0.002917
\end{verbatim}

\begin{Shaded}
\begin{Highlighting}[]
\KeywordTok{tidy}\NormalTok{(GPAlm)}
\end{Highlighting}
\end{Shaded}

\begin{verbatim}
## # A tibble: 2 x 5
##   term        estimate std.error statistic       p.value
##   <chr>          <dbl>     <dbl>     <dbl>         <dbl>
## 1 (Intercept)   2.11      0.321       6.59 0.00000000130
## 2 ACT           0.0388    0.0128      3.04 0.00292
\end{verbatim}

\textbf{Decision}

\begin{Shaded}
\begin{Highlighting}[]
\KeywordTok{qt}\NormalTok{(}\FloatTok{0.975}\NormalTok{,}\DecValTok{118}\NormalTok{)}
\end{Highlighting}
\end{Shaded}

\begin{verbatim}
## [1] 1.980272
\end{verbatim}

\begin{itemize}
\tightlist
\item
  \textbf{Critical Value}: t\textasciitilde{}(118,0.975) = 1.980272
\item
  \textbf{Decision Rule}:

  \begin{itemize}
  \tightlist
  \item
    Reject H0: if \textbar{}t\textbar{} \textgreater{} t(118,0.975)
  \item
    Fail to reject H0:\textbar{}t\textbar{} \textless{} t(118,0.975)
  \end{itemize}
\end{itemize}

\textbf{Conclusion}

\begin{itemize}
\tightlist
\item
  For the current data, we have test stats of 3.040 and p-value of
  2.916604e-03 for slope. Since t\_stats:3.040 \textgreater{} critical
  value1.980272, we reject the null and conlcude that there is
  significant linear association exists between student's ACT score (X)
  and GPA at the end of the freshman year (Y).
\end{itemize}

\hypertarget{part-2}{%
\subsubsection{part 2}\label{part-2}}

\[ GPA(estimated) = 2.11405 + 0.03883ACT \]

\hypertarget{part-3}{%
\subsubsection{part 3}\label{part-3}}

\textbf{95\% Confidence Interval for slope}

\[slope +(-) t(118,0.975)*se(beta1)\]

Calculation by r

\begin{Shaded}
\begin{Highlighting}[]
\KeywordTok{confint}\NormalTok{(GPAlm,}\DataTypeTok{level=}\FloatTok{0.95}\NormalTok{)}
\end{Highlighting}
\end{Shaded}

\begin{verbatim}
##                  2.5 %     97.5 %
## (Intercept) 1.47859015 2.74950842
## ACT         0.01353307 0.06412118
\end{verbatim}

0.03883 ± 1.980272*0.01277302 = (0.01353595,0.06412405)

\textbf{Conclusion}

\begin{itemize}
\item
  With 95\% confidence, we estimate that the mean GPA increases by
  somewhere between 0.01353595 and 0.06412405 for each additional point
  in ACT.
\item
  the interval does not conclude 0. The director of admissions might be
  interested in whether the confidence interval includes zero because
  they are interested in learning about whether a higher ACT scores can
  be a potential predictor for higher gpa, in this case, it can be a
  potential predictor.
\end{itemize}

\hypertarget{part-4}{%
\subsubsection{part 4}\label{part-4}}

\begin{Shaded}
\begin{Highlighting}[]
\NormalTok{new_data <-}\StringTok{ }\KeywordTok{data.frame}\NormalTok{(}\DataTypeTok{ACT=}\KeywordTok{c}\NormalTok{(}\DecValTok{28}\NormalTok{))}
\KeywordTok{predict}\NormalTok{(GPAlm, }\DataTypeTok{newdata=}\NormalTok{new_data, }\DataTypeTok{interval=}\StringTok{"confidence"}\NormalTok{, }\DataTypeTok{level=}\FloatTok{0.95}\NormalTok{)}
\end{Highlighting}
\end{Shaded}

\begin{verbatim}
##        fit      lwr      upr
## 1 3.201209 3.061384 3.341033
\end{verbatim}

\begin{itemize}
\tightlist
\item
  \textbf{Interpretation}: For people who have ACT score = 28, the
  expected mean GPA score can vary between 3.061384 and 3.341033.
\end{itemize}

\hypertarget{part-5}{%
\subsubsection{part 5}\label{part-5}}

\begin{Shaded}
\begin{Highlighting}[]
\KeywordTok{predict}\NormalTok{(GPAlm, }\DataTypeTok{newdata=}\NormalTok{new_data, }\DataTypeTok{interval=}\StringTok{"prediction"}\NormalTok{, }\DataTypeTok{level=}\FloatTok{0.95}\NormalTok{)}
\end{Highlighting}
\end{Shaded}

\begin{verbatim}
##        fit      lwr      upr
## 1 3.201209 1.959355 4.443063
\end{verbatim}

\begin{itemize}
\tightlist
\item
  \textbf{Interpretation}: For a new person with ACT score of 28, her or
  his mean GPA score can vary between 1.959355 to 4.443063
\end{itemize}

\hypertarget{part-6}{%
\subsubsection{part 6}\label{part-6}}

\begin{itemize}
\tightlist
\item
  prediction interval is wider than the confidence interval because the
  prediction interval need to account for an additional error term
  whereas confidence interval don't.
\item
  For confidence interval, we are using expected mean GPA for all ACT
  that equal to 28, but in prediciton, we are looking at the range for
  new people with ACT of 28, so prediciton have more error involved.
\end{itemize}

\end{document}
